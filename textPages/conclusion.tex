\chapter{Conclusion}
\label{conclusion}

The \ac{PoC} demonstrated the feasibility of integrating an \ac{LLM} into an achievement evaluation process, leveraging dynamic questionnaires and predefined criteria for robust assessments. 
This chapter summarizes the findings and outlines potential next steps to further refine and expand the implementation.

\section{Findings}
The development and testing of the \ac{PoC} yielded several key findings:
\begin{itemize}
    \item The integration of the \ac{LLM} with dynamic category-based questionnaires effectively streamlined the evaluation process, providing actionable recommendations with detailed explanations.
    \item Predefined inclusion and exclusion criteria stored in \texttt{.csv} files were instrumental in ensuring consistent evaluations aligned with organizational goals.
    \item Prompt engineering played a critical role in refining the \ac{LLM}'s responses. Iterative improvements to the system prompt resulted in more accurate evaluations of achievements.
    \item The web-based prototype provided a user-friendly interface, demonstrating how end-users could easily interact with the system to submit achievements for assessment.
    \item Testing identified the impact of user answers, category-specific criteria, and prompt settings on the evaluation outcomes, highlighting areas for further optimization.
\end{itemize}

\section{Next Steps}
Building on the success of the \ac{PoC}, the following steps are recommended for advancing this project:

\subsection*{Expanding Criteria}
Additional inclusion and exclusion criteria should be added to the \texttt{.csv} files to cover a broader range of categories and scenarios. 
This will improve the versatility of the system and ensure it meets diverse organizational needs.

\subsection*{Integration into UI5 and CAP}
The current web-based prototype needs to be adapted into a full-fledged implementation using SAP UI5 and CAP to integrate seamlessly with the existing corporate website. 
This transition will align the solution with the organization's technology stack and enhance scalability.

\subsection*{Refining Prompt Engineering}
Further iterations of prompt engineering should be conducted to maximize the accuracy and reliability of the \ac{LLM}'s evaluations. 
This includes experimenting with different temperature settings to determine the optimal balance between creativity and consistency in responses.

\subsection*{Comprehensive Testing}
Additional testing is necessary to validate the system across various edge cases and categories. 
This includes stress testing the application with large datasets and diverse achievements to ensure consistent performance and reliability.

\subsection*{User Feedback and Iterative Improvement}
Incorporating feedback from end-users will be vital for refining the system's usability and effectiveness. 
Iterative improvements based on user insights will ensure the solution aligns with real-world expectations and requirements.

These next steps will transform the \ac{PoC} into a robust and scalable system, providing a valuable tool for achievement evaluation and recognition within the organization.
