\chapter{Introduction}
\label{introduction}
\nocite{*}

\section{Background}
The \textit{Champions Circle} software at SAP Labs India is an innovative platform that recognizes and rewards employees for exceptional achievements. 
By fostering a culture of appreciation, it ensures that accomplishments across various domains, such as innovation, leadership, and diversity, are celebrated. 
However, as the software's usage grows, so does the need for an efficient and streamlined recommendation process.

The existing workflow involves multiple approval levels: from the manager, to the Leader, and finally to a jury that determines the worthiness of the achievement for the award. 
This manual process, while thorough, can be time-consuming and subject to inconsistencies. 
To address this, a \ac{PoC} was developed to integrate a \ac{LLM} extension into the \textit{Champions Circle} platform. 
This extension evaluates recommendations at the initial stage, providing a score and explanatory feedback to assist managers in their decision-making.

This paper details the \ac{PoC} development process, including the creation of a test environment to mimic the \textit{Champions Circle} software. 
This environment leveraged technologies such as \textit{Vite}, \textit{Node.js}, and \textit{HTML/JavaScript} while using SAP's categories and criteria for validation.

\section{Preread}
Before embarking on the \ac{PoC} development, it was crucial to understand the existing \textit{Champions Circle} framework and its operational workflow. 
This involved reviewing documentation shared by SAP Labs India. 

Additionally, I was provided with multiple SharePoint pages that detailed the inclusion and exclusion criteria for each category. 
I studied these criteria and simplified them into a \texttt{.csv} file, which was then used as extra information for the \ac{LLM}.

By combining insights from the \textit{Champions Circle} system and advancements in \ac{LLM} technology, the \ac{PoC} aims to enhance the recommendation process, 
ensuring fairer and more efficient evaluations.